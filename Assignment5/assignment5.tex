\documentclass{article}
\title{Assignment 5}
\author{Viktor Nonov}
\date{March 19, 2018}
\usepackage[cm]{fullpage}
\usepackage{enumerate}
\usepackage{amsmath}
\usepackage{amssymb}

\begin{document}
\section*{Assignment 5 Solutions}

\section{}
a. $(\exists x \in \mathbb{N})[x^3 = 27]$ \\
b. $(\exists x \in \mathbb{N})[x > 1 000 000]$ \\
c.
Another way of saying it would: There exists a natural number $n > 1$ that is not prime. \\
Let P(x) be the "not prime property of x" \\
$(\exists(x > 1 \wedge x \in N))[P(x)]$ \\
$(\exists x \in \mathbb{N})[n > 1 \wedge P(x)]$ - looks better\\
We know that the set $\mathbb{N}$ starts with $\{1,2,3....\}$, so we can rephrase it as: \\
"There exists a natural number $n \neq 1$ that is not prime." \\
$(\exists x \in \mathbb{N})[n \neq 1 \wedge P(x)]$ \\

\section{}
a. $(\forall x \in \mathbb{N})[x^3 \neq 28]$ \\
b. $(\forall x \in \mathbb{N})[x > 0]$ \\
c. Let P(x) be the "prime property of x" \\
$(\exists x \in \mathbb{N})[x > 1 \wedge P(x)]$ \\

\section{}
let P be all people. \\
a. $(\forall p1 \in P)(\exists p2 \in P) [ loves(p1, p2) ]$ \\
b. $(\forall p \in P)[ tall(p) \vee short(p) ]$ \\
c. $(\forall p \in P)[ tall(p) ] \vee (\forall p \in P) [ short(p) ]$ \\
d. $(\forall p \in P)[ \neg athome(p) ]$ \\
e. $(\exists p \in P)[ p = John \wedge comes(p) ] \implies (\forall p \in P)[woman(p) \implies Leave(p)]$ \\
f. $(\forall p \in P)[man(p) \wedge comes(p)] \implies (\forall p \in P)[woman(p) \implies Leave(p)]$ \\

\section{}
a. $(\exists x \in \mathbb{R})(\forall a \in \mathbb{R})[x^2 + a = 0]$ \\
b. $(\exists x \in \mathbb{R})(\forall a \in \mathbb{R})[a < 0 \implies x^2 + a = 0]$\\
c. $(\forall x \in \mathbb{R})[rational(x)] \Leftrightarrow (\forall x \in \mathbb{R})(\exists p,q \in \mathbb{N})[x = p/q]$\\
d. $(\exists x)[irational(x)] \Leftrightarrow (\exists x)(\forall p,q \in \mathbb{N})[x \neq p/q]$\\
e. Rephrasing to "for each irrational number m there is an irrational number n so $n > m$" \\
$(\forall m \in \mathbb{R} - \mathbb{Q})[\exists n \in \mathbb{R} - \mathbb{Q}](n > m)] \Leftrightarrow (\forall m)(\exists n)(\forall p,q \in \mathbb{N})[m \neq p/q \wedge n \neq p/q \wedge n > m]$ \\

\section{}
C - all cars \\
x has property D(x) representing domestic cars \\
x has property M(x) representing badly made cars \\
a. $(\forall x \in C)[D(x) \implies M(x)]$ \\
b. $(\forall x \in C)[\neg D(x) \implies M(x)]$ \\
c. $(\forall x \in C)[M(x) \implies D(x)]$ \\
d. $(\exists x \in C)[D(x) \wedge \neg M(x)]$ \\
e. $(\exists x \in C)[\neg D(x) \wedge M(x)]$ \\

\section{}
$\forall m,n \in \mathbb{R}$ we have a x with property Q(x), such that $m < x < n$ \\
$(\forall m,n \in \mathbb{R})(\exists x)[Q(x) \wedge (m < x < n)]$ \\

\section{}
Given: \\
all people - P \\
person - p \\
time - t \\
person with my default property fool(p,t) $\Leftrightarrow (\exists p \in P)[Viktor]$ \\

Split the sentece into:

You may fool "all the people some of the time", / $(\forall p \in P)(\exists t)[fool(p,t)]$ \\
you can even "fool some of the people al the time", / $(\forall t)(\exists p \in P)[fool(p, t)]$\\
but you cannot "fool all the people all the time". / $(\forall t)(\forall p \in P)[fool(p,t)]$ \\

$(\forall p \in P)(\exists t)[fool(p,t)] \vee (\forall t)(\exists p \in P)[fool(p, t)] \wedge \neg((\forall t)(\forall p \in P)[fool(p,t)]) \Leftrightarrow$ \\
$(\forall p \in P)(\exists t)[fool(p,t)] \vee (\forall t)(\exists p \in P)[fool(p, t)] \wedge (\exists t)(\exists p \in P)[\neg fool(p,t)])$

\section{}
a driver is involved in an accident every 6 seconds \\
a driver - d \\
6 seconds - t \\
driver has a property accident - A(d, t) \\
$(\exists d)(\forall t)[A(d, t)]$ \\

\section{}
$(\forall t)(\exists d)[A(d, t)]$ \\

\end{document}
